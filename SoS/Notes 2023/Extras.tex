\mychapter{100}{Addendum}
	\begin{tcolorbox}[title=Recursive algorithm for Extended Eucleadian Theorem]\label{add:codeForRecursive}	
	The \hyperref[theo:extendedEuclid]{extended Eucleadian Theorem} helps us find modular inverses in linear time. Here is a small code snippet showing that using recursion.
		\begin{lstlisting}[style=CStyle]
	// C++ program to find multiplicative modulo
	// inverse using Extended Euclid algorithm.
	#include <iostream>
	using namespace std;

	// Function for extended Euclidean Algorithm
	int gcdExtended(int a, int b, int* x, int* y);

	// Function to find modulo inverse of a
	void modInverse(int a, int m)
	{
		int x, y;
		int g = gcdExtended(a, m, &x, &y);
		if (g != 1)
			cout << "Inverse doesn't exist";
		else
		{
			
			// m is added to handle negative x
			int res = (x % m + m) % m;
			cout << "Modular multiplicative inverse is " << res;
		}
	}

	// Function for extended Euclidean Algorithm
	int gcdExtended(int a, int b, int* x, int* y)
	{
		
		// Base Case
		if (a == 0)
		{
			*x = 0, *y = 1;
			return b;
		}
		
		// To store results of recursive call
		int x1, y1;
		int gcd = gcdExtended(b % a, a, &x1, &y1);

		// Update x and y using results of recursive
		// call
		*x = y1 - (b / a) * x1;
		*y = x1;

		return gcd;
	}

	// Driver Code
	int main()
	{
		int a = 3, m = 11;

		// Function call
		modInverse(a, m);
		return 0;
	}

	// This code is contributed by khushboogoyal499

		\end{lstlisting}
	\end{tcolorbox}