% !TEX TS-program = pdflatex
% !TEX encoding = UTF-8 Unicode

% This is a simple template for a LaTeX document using the "article" class.
% See "book", "report", "letter" for other types of document.

\documentclass[11pt]{article} % use larger type; default would be 10pt

\usepackage[utf8]{inputenc} % set input encoding (not needed with XeLaTeX)
\usepackage{authblk}  %For extra authors
\usepackage{longtable}
\usepackage{multirow}
\usepackage{siunitx}
\usepackage{booktabs}
\usepackage{makecell}
\usepackage{caption}

%%% Examples of Article customizations
% These packages are optional, depending whether you want the features they provide.
% See the LaTeX Companion or other references for full information.

%%% PAGE DIMENSIONS
\usepackage{geometry} % to change the page dimensions
\geometry{a4paper} % or letterpaper (US) or a5paper or....
\geometry{margin=1in} % for example, change the margins to 2 inches all round
% \geometry{landscape} % set up the page for landscape
%   read geometry.pdf for detailed page layout information

\usepackage{graphicx} % support the \includegraphics command and options

% \usepackage[parfill]{parskip} % Activate to begin paragraphs with an empty line rather than an indent

%%% PACKAGES
\usepackage{booktabs} % for much better looking tables
\usepackage{array} % for better arrays (eg matrices) in maths
\usepackage{paralist} % very flexible & customisable lists (eg. enumerate/itemize, etc.)
\usepackage{verbatim} % adds environment for commenting out blocks of text & for better verbatim
\usepackage{subfig} % make it possible to include more than one captioned figure/table in a single float
% These packages are all incorporated in the memoir class to one degree or another...

%%% HEADERS & FOOTERS
\usepackage{fancyhdr} % This should be set AFTER setting up the page geometry
\pagestyle{fancy} % options: empty , plain , fancy
\renewcommand{\headrulewidth}{0pt} % customise the layout...
\lhead{}\chead{}\rhead{}
\lfoot{}\cfoot{\thepage}\rfoot{}

%%% SECTION TITLE APPEARANCE
\usepackage{sectsty}
\allsectionsfont{\sffamily\mdseries\upshape} % (See the fntguide.pdf for font help)
% (This matches ConTeXt defaults)

%%% ToC (table of contents) APPEARANCE
\usepackage[nottoc,notlof,notlot]{tocbibind} % Put the bibliography in the ToC
\usepackage[titles,subfigure]{tocloft} % Alter the style of the Table of Contents
\renewcommand{\cftsecfont}{\rmfamily\mdseries\upshape}
\renewcommand{\cftsecpagefont}{\rmfamily\mdseries\upshape} % No bold!
\setlength\parindent{24pt}

%%% END Article customizations

%%% The "real" document content comes below...

\title{\huge{Deciphering the Mysteries} \\ \normalsize Plan of Action- SoS 2023(Cryptography)}
\author{Arya Vishe \\21D070018, IIT-B} % auteur
\affil{Mentor- Nilabha Saha}


\date{20/06/23} % Activate to display a given date or no date (if empty),
         % otherwise the current date is printed 


\begin{document}
\maketitle

\section{Objectives}

Over the period of the next few weeks, I intend to start off with classical ciphers such as Vigen\'{e}re Ciphers. After that, I can spend a few weeks to build up some knowledge on fields such as Number theory, Group theory, and Abstract Algebra, as a basis for modern cryptography techniques such as RSA, DES, AES encryptions, and, if time permits, Stream and Block ciphers, Hash functions, Pseudo-random bit generation and Elliptic curves.  \par
I will also be having a project running in parallel to SoS so I will be keeping relatively low loads in the beginning.

\section{Outline of the timetable}

\captionsetup{labelformat=empty}

\begin{center}
\begin{longtable}{m{0.20\linewidth}|m{0.75\linewidth}}\\
\multicolumn{2}{c}%
{\tablename\ \thetable\ -- \textit{}} \\

\multicolumn{2}{l}{\textit{}} \\
\endfoot
\hline
\endlastfoot
\textbf{Week 1} (29th May- 4th June) & Overview and introduction to cryptography\footnotemark, Probabilistic cryptanalysis of Vigen\'{e}re ciphers \\  \\
	\textbf{Week 2} (5th June- 11th June) & Probability theory, Number theory, Information theory, Abstract Algebra \\  \\
	\textbf{Week 3} (12th June- 18th June) &  Generation of Pseudorandom bits, Stream and Block Ciphers \\ \\
	\textbf{Week 4} (19th June- 25th June) & \\ \\
	\hline \hline \\
	\textbf{Mid-summer Report week}\footnotemark (26th June- 2nd July) & Lag week\\ \\
	\hline \hline \\
	\textbf{Week 6} (3rd July- 9th July) & Diffie Heilman (Discrete Logarithm), RSA (Factorisation) \\ \\
	\textbf{Week 7} (10th July- 16th July) & Finite Fields, Data Encryption Standard (DES), Advanced Encryption Standard (AES) \\  \\
	\textbf{Week 8} (17th July- 23rd July) & Hash functions, Authentication problem, Digital signatures \\ \\
	\hline \hline
	\textbf{Final Project submission\footnotemark} (24th July- 30th July) & Elliptic curves\\  
	\hline \hline
\end{longtable}
\end{center}
 
\addtocounter{footnote}{-2}
\footnotetext{From both \cite{Silverman} and \cite{Holden}}
\addtocounter{footnote}{1}
\footnotetext{Tentative date- 15th June}
\addtocounter{footnote}{1}
\footnotetext{Tentative date- 15th July}

\begin{thebibliography}{20} 
 \bibitem[1]{Silverman} Jeffrey Hoffstein, Jill Pipher, Joseph H. Silverman, \emph{An Introduction to Mathematical Cryptography}
 \bibitem[2]{Holden} Joshua Holden, \emph{THE MATHEMATICS OF SECRETS}
 \bibitem[3]{Modern} Jonathan Katz and Yehuda Lindell, \emph{Introduction to Modern Cryptography}
 \bibitem[4]{Vanstone} Alfred J. Menezes, Paul C. van Oorschot, Scott A. Vanstone, \emph{HANDBOOK of APPLIED CRYPTOGRAPHY}
 \bibitem[5]{CodeBook} Simon Singh,\emph{ "The Code Book- HOW TO MAKE IT, BREAK IT, HACK IT, CRACK IT"}
\end{thebibliography}

\noindent{\textbf{\underline{Note}}- \cite{Silverman} and \cite{Holden} are going to be referred to primarily, \cite{Modern} will be used for Modern methods and \cite{Vanstone} will be used as an additional reference. \cite{CodeBook} is merely for introductory puposes}
\end{document}
