\mychapter{0}{Author's Note}\label{chap:Auth_note}
	
	\begin{tcolorbox}[colback=MustardAddicted!5!white,colframe=MustardAddicted!75!black]
		\small Hello! \vspace{0.5cm}
		Here are a few things which I would like you, the reader, to know before you start with the report-
		\begin{itemize}
			\item This report is written as a beginner's guide to Cryptography. I have tried to ensure that the initial learning curve is gentler. In doing so, a bit of mathematical rigour and some technical aspects are glossed over. In a lot of places, informal language was also used to break up some breathing space in between all of the jargon.
			\item \cite{Silverman} was a heavy influence in writing this report. I have mostly stuck to the format of the reference textbook in the later sections of this report but if time permits, I would like to explore some more topics outside of the actual book.
			\item The colour \hyperref[chap:Auth_note]{``teal''} has exclusively been used for hyperlinks to ensure that the reader is able to read the further sections without having to skim through every concept discussed before that. (Only the table of contents has hidden hyperlinks)
			\item It is highly recommended that the reader goes through the first chapter (\emph{Mathematical Base to Cryptography}) before continuing to other sections. But it is understandable that s/he might get bored by just theoritical knowledge without seeing any link to cryptography so they can free go to any of the further sections. Wherever possible, I have added hyperlinks, in the later chapters, to the earlier concepts to ensure that the first chapter can act as a look-up table for readers.
		\end{itemize}
	\end{tcolorbox}

	\mysection{1}{Acknowledgements}
		This work was written as a part of the Summer Of Science, 2023 by the MnP Club, IIT Bombay. I read in its entirety \cite{CodeBook} to get a surface level introduction to the topic and then mostly followed the reference book \cite{Silverman} and referred to online resoruces like \cite{Hypr} for a deeper dive. \par Some of the sections in the report are heavily inspired by the above resources and at several points might simply be a paraphrasing of the reference books. Nevertheless, I have tried to compress the contents of these sources wherever needed, explored some minor topics on my own to a deeper depth, and tried my best to break down too much of difficult mathematics into easier chunks while still preserving sufficient depth. \par I would like to acknowledge the help of my mentor, Nilabha Saha, for his help and support. He was extremely helpful and friendly and his enthusiasm for the topic definitely rubbed off on me and gave me enough encouragement to devote the hours I have put into this report. He cleared my doubts at every stage at the latest and that allowed me to get through some of the most difficult sections of this report.\par Some topics in the end had to be skipped as they were only explored partially and so I opted to remove them completely in order not to ruin the structure of this report.\normalsize